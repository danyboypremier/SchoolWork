\documentclass[12pt]{article}
\usepackage[frenchb]{babel}
\usepackage[utf8]{inputenc}% encodage utf-8
\usepackage{verbatim}% permet de faire des sections de comments
\usepackage{url}% permet d'écrire une adresse sans backslash
\usepackage[document]{ragged2e}% commandes pour texte justifié, centré, etc.
\usepackage{geometry}% changer les bordures
    \geometry{
    top=25mm,
    }
\renewcommand{\contentsname}{Table des Matières}
\renewcommand{\refname}{Référence}
\newcommand{\HRule}{\rule{\linewidth}{0.5mm}}
\frenchbsetup{ItemLabels=\textbullet}

% début du document
\begin{document}

% page titre
\begin{titlepage}
\begin{center}
\textsc{\LARGE MET5311}\\[1.5cm]
\textsc{\Large Livrable 3}\\[0.5cm]
\HRule \\[0.4cm]
{ \huge \bfseries Retour d'expérience \\[0.4cm] }
\HRule \\[1.5cm]
\noindent
\begin{minipage}[t]{0.4\textwidth}
\begin{flushleft} \large
\emph{Auteur:}\\
Dany \textsc{Deroy}\\
\emph{Code permanent:}\\
DERD92988493
\end{flushleft}
\end{minipage}
\begin{minipage}[t]{0.4\textwidth}
\begin{flushright} \large
\emph{Enseignant:} \\
Régis \textsc{Barondeau}\\
\emph{Université:} \\
Université du Québec à Montréal
\end{flushright}
\end{minipage}
\vfill
{\large \today}
\end{center}
\end{titlepage}
\newpage

% table des matières
\tableofcontents
\newpage
\justify

\begin{comment}
Chaque étudiante, étudiant rédigera son retour d'expérience sur l'ensemble du cours en 3000 mots +ou- 		5% en répondant aux questions suivantes :
\end{comment}

% début du texte
\section{I - Outils}
\hrulefill

\subsection{Que pensiez-vous des collecticiels avant le cours?}
\paragraph{}
J'ai l'avantage d'étudier dans un domaine où les outils de collaboration font partie de la culture. Je parle ici de logiciel tel que GitHub (un dérivée de Git) ou encore BitBucket. La majorité des travaux textuels se font à l'aide de Google Docs pour justement nous permettre d'être un peu n'importe où et être capable de contribuer au travail. J'ai toujours comblé mes besoins à l'utilisation immédiate et non nécessairement en faisant de la recherche. Je veux dire par là que je me procure un logiciel en fonction de ce que je veux faire et non faire un projet en fonction de ce que je connais déjà comme outil. Par exemple, j'ai à faire un schéma style UML en équipe, donc je cherche un logiciel qui me permet de le faire en collaboration.

\subsection{Que pensez-vous des collecticiels après le cours?}
\paragraph{}
Je connaissais déjà l'utilité, par contre ça m'a permis d'approfondir ma bibliothèque d'outils collaboratifs. Je crois vraiment que l'avenir est dans la synchronisation des individus pour réaliser les objectifs. Le monde est tellement en mouvement qu'il devient difficile de faire des rencontres face à face, mais ça ne devrait plus être un frein à l'évolution d'un projet commun.

\subsection{Que voyez-vous en regardant Quora?}
\paragraph{}
Simplicité! Côté design, l'organisation semble être faite pour inciter l'utilisateur à consulter plusieurs articles selon son champs d'intérêt. Il y a de nouveaux articles toutes les secondes!

\subsection{Que pensez-vous de Quora?}
\paragraph{}
Ce que je trouve intéressant, c'est le sérieux, du moins en apparence, des utilisateurs. Je n'ai pas testé avec une question, je ne peux donc parler du temps de réponse, mais il semble facile de chercher à l'intérieur du site. Je suis impressionné par le nombre de sujets. Je crois même participer prochainement aux discussions qui m'intéresse et l'utiliser comme source d'information.

Je suis impressionné par le nombre de sujets disponibles (questions) et à quel rythme il y a des réponses. Les gens semblent dévoués à la communauté.

\subsection{Qu’est-ce que Quora éveille comme questions?}
\paragraph{}
Le sérieux des utilisateurs. Faut se demander si nous devons toujours faire confiance aux réponses. C'est relativement facile de faire à croire n'importe quoi en sachant que généralement les gens qui cherchent une réponse sont susceptibles d'être corrompus. Je ne crois pas que ce doit être la seule source d'information. Je crois qu'il faut surtout mettre de l'emphase sur comment chercher l'information et la traiter.

Si je comprends bien, les gens atteignent une certaine renommé en étant ``upvoté``. Ça peut nous aider à distinguer les personnes sérieuses des ``Troll``.

Ce genre de site peut amener les gens à s'intéresser à certains sujet, mais peut avoir aussi l'effet inverse lorsque nous n'avons plus à chercher l'information, juste à la demander.

\newpage
\section{II - Individu/équipe/groupe}
\hrulefill

\subsection{Qu’avez-vous ressenti la première fois que vous avez utilisé Evernote, Google Hangouts et Confluence? Comment a évolué ce ressenti?}
\subsubsection{Evernote}
\paragraph{}
En commençant par Evernote, je trouvais l'idée géniale et je voyais déjà comment l'outil pouvait m'être utile. Mais même avec les plus grandes intentions du monde, je suis incapable de l'utiliser dans mes activités quotidiennes. Je crois que c'est de changer mes habitudes le problème. Mon mécanisme de recherche est trop centré sur Google. Ce n'est même pas à cause de l'application manquante sur les systèmes Linux qui m'embête, car l'application web est quand même utilisable. C'est vraiment difficile pour moi de changer des habitudes qui dates d'au moins 15 ans.

\subsubsection{Google Hangouts}
\paragraph{}
Hangouts était déjà un logiciel que j'utilisais, par contre je n'avais jamais utilisé la fonction de vidéo conférence. Même si je suis quelqu'un de techno et très débrouillard, j'admets que ce n'est pas le logiciel le plus simple à utiliser. Nous en avons eu la preuve lors du cours donné sur celui-ci. Honnêtement, je ne crois pas l'utiliser de cette façon prochainement. J'ai de la difficulté à me trouver des arguments pour mon utilisation qui me forcerait à utiliser autre chose que le logiciel Skype.

\subsubsection{Confluence}
\paragraph{}
Confluence...l'idée est bonne, mais difficile d'avoir de l'agrément à l'utiliser. Je mets tout ça sur mon manque d'expérience. Je n'étais pas motivé à faire quelque chose de beau. Efficace oui, mais pas nécessairement esthétique.

\subsection{Que pensiez-vous du rôle de la coordination dans l’usage d’un collecticiel avant le cours?}
\begin{comment}
- 3 idées (développez)
- 2 questions, 1 analogie**
** "Une analogie est un processus de pensée par lequel on remarque une similitude de forme entre deux 		choses, par ailleurs de différentes natures ou classes." Source : Wikipedia page consultée le 13.01.2015
\end{comment}

\subsubsection{3 idées à développer}
\paragraph{}
Au départ je ne voyais pas vraiment la coordination comme étant un élément important pour l'utilisation d'un collecticiel. Je me disais que justement, si on parle d'un logiciel comme Google docs, les gens investissent leur temps comme ils veulent et de la façon qu'ils le veulent.

\paragraph{}
Pour moi, la coordination se limitais à un minimum d'organisation: qui fait quoi et avec quoi nous le faisons. Généralement un 15 minutes suffisait.

\paragraph{}
Par expérience, la coordination a presque toujours été dirigée par une personne du groupe, dit comme le leader, où c'est lui qui traçait les lignes directrices; attribution des tâches, planification des échéanciers, etc. 

\subsubsection{2 questions}
\paragraph{}
Quels sont les outils permettant une meilleure coordination?

\paragraph{}
À quel individu ou à quel groupe d'individu devraient avoir la charge de coordonner?

\subsubsection{1 analogie}
\paragraph{}
Le meilleure exemple pour exprimer mes idées et mes questions est le milieu où je travaille. J'ai été plus de 14 ans dans le milieu de la restauration et j'ai vu pas mal tout de; coordination, organisation, petite à grande équipe. Nous parlons d'avoir les gens au bon poste, avec l'objectif à atteindre et les bons outils pour y parvenir. Il faut toujours avoir les bonnes personnes responsables du plan de match.

\subsection{Que pensez-vous du rôle de la coordination dans l'usage d'un collecticiel après le cours?}
\begin{comment}
- 3 idées (développez)
- 2 questions
- 1 analogie
Expliquez comment vos nouvelles réponses sont reliées à vos réponses initiales
\end{comment}

\subsubsection{3 idées à développer}
\paragraph{}
La coordination est le début de l'atteinte des objectifs. Sinon, comment les atteindre si tous les gens tirent de leur côté ou vont dans toutes les directions?

\paragraph{}
Je crois encore que la coordination doit être dirigée par une personne ou un très petit groupe pour éviter de faire du surplace et gérer plsu facilement les conflits.

\paragraph{}
Les outils doivent être expérimentés avant de décider lequel est le mieux adapté. Il ne faut pas s'attendre à régler tous les problèmes avec un outil qui fait tout, mais nous pourrons ajuster nos attentes en conséquence.

\subsubsection{2 questions}
\paragraph{}
Si nous parlons de collecticiel, il y toujours un peu de technologie à l'arrière, Donc, est-ce à la porté de tous d'être capable de maitriser l'outil? Pensez à votre mère, serait-elle capable d'utiliser un logiciel tel que Hangouts?

\paragraph{}
Qu'elle est l'effort qui doit être mis pour changer les habitudes des gens? Est-ce vraiment rentable pour certaines situations? Comemnt gérer la peur du changement?

\subsubsection{1 analogie}
\paragraph{}
Je peux encore revenir sur l'exemple d'un restaurant. Il y a toujours une courbe d'apprentissage lorsqu'un nouvel outil arrive et il est encore plus difficile de changer les habitudes, les techniques lorsque les gens ont une certaine expérience. Je veux dire que certaines personnes vont être attachées à leurs idées puisqu'ils ont acquis une reconnaissance par leur ancienneté.

\newpage
\section{III - Mise en pratique}
\hrulefill

\subsection{Donnez un exemple concret de coordination, de coopé-ration et de collaboration en ligne sur la base des défi-nitions vues en classe}

\subsubsection{Coordination}
\paragraph{}
Choisir le ou les bons outils. Exemple, choisir l'outil de communication tel que Hangouts, Skype, etc. Choisir l'outil de production comme Google docs pour du texte, Google drawing pour la schématisation, etc.

Prenons en exemple la production du livrable 1. Quoi que le logiciel pour le stockage des sources était prédéfini (Evernote), nous avons choisi le logiciel Skype pour partager en audio.

\subsubsection{Coopération}
\paragraph{}
Investir du temps dans l'atteinte de l'objectif commun.

Prenons en exemple le livrable 2, soit ``avant`` le développement de la thèse commune sur Confluence. Chaque parti développe ses idées pour l'attente des objectifs.

\subsubsection{Collaboration}
\paragraph{}
Mettre en commun ses idées.
Prenons en exemple le livrable 2, soit le développement de la thèse commune sur Confluence. Chaque parti apporte ses idées pour enrichir le travail. Chaque personne est autonome et peut, soit ajouter de nouvelles idées ou encore faire grandir celles déjà existantes.

\subsection{Quelles seraient les contraintes liées à l’implantation\\ d’un Intranet utilisant un moteur wiki dans votre organisation?}
\paragraph{}
Toujours en prenant l'exemple de mon milieu de travail. Si nous restons au niveau des employés de cuisine, le développement d'idées individuelles est extrêmement fermé. C'est difficile d'apporter des solutions à un problème sans passer par toute la hiérarchie. Alors juste de penser à implanter un wiki pour exemple des trucs ou des méthodes de travail est déjà un problème en soit.

\subsection{On vous charge d’organiser et d’animer une réunion en ligne regroupant 15 personnes.\\Comment préparez-vous la réunion?}
\begin{itemize}
    \item Je choisis une liste de 3 ou 4 logiciels de conférence à la Hangouts.
    \item J'en choisis 1 où la majorité est d'accord.
    \item J'établis un plan B au cas où le premier logiciel nous ferait défaut.
    \item J'établis un plan C de dernier recours, comme un chat fiable.
    \item Si possible je teste et je demande aux gens de tester (peut-être faire une pré-séance)
\end{itemize}

\subsection{Mise en situation}
\paragraph{}
Un représentant de la compagnie SuperApp vous présente leur dernière solution collaborative en gestion de projet pour les PME. La solution est accessible en ligne depuis le  Cloud  de la compagnie, elle fonctionne sur PC et Mac ainsi que sur iOS, Android, Windows Phone et BlackBerry OS. Quelles questions allez-vous poser au représentant en lien avec la sécurité et la protection de la vie privée? Pourquoi?

\begin{itemize}
    \item Je demande quel type d'encryptage et si l'utilisation des cookies est obligatoire\\
Je veux savoir si des mesures sont prises pour éviter l'interception des données et aussi le stockage d'information local par les cookies.
    \item Je veux aussi savoir dans quel pays les serveurs sont installés.\\
Pour savoir quels sont les lois en vigueur dans le pays en question.
    \item Je veux savoir si mes données peuvent être vendues ou partagées avec une tierce partie.
\end{itemize}

\newpage
\section{Conclusion}
\hrulefill
\paragraph{}
Quoi retenir de ce cours? Que l'évolution des outils de collaboration sont en plein expansion. Il devient donc important d'être capable d'analyser nos besoins pour être capable d'identifier le ou les outils qui les combleront.

\paragraph{}
Quelques petits commentaire sur le cours :\\
Il aurait été intéressant d'avoir les deux premiers livrables plus tôt dans la session. Et possiblement axer le dernier livrable plus sur l'analyse de nos nouvelles connaissances apprises en suivant ce cours. Les question du dernier livrable me semble quelque peu ambigu.

\paragraph{}
J'aimerais faire une petite note sur la façon d'on ce document a été créé, soit avec \LaTeX, un espèce de code qui permet la mise en page uniforme sur n'importe quel format.

Vous trouverai le code source à l'adresse suivante :\\
\url{https://github.com/danyboypremier/MET5311Retour_Experience.git}

Un lien vers la documentation \LaTeX\ :\\
\url{http://latex-project.org/}

Puisqu'il s'agit d'un texte en format ``texte``, il est donc possible de le placer sous gestion de source. Il devient alors possible de garder un historique des changements.
\newpage

% bibliographie
\begin{thebibliography}{99}
\bibitem{bib:website} MET 5311 Hiver 2015, Cours 1\\
    \url{http://www.moodle2.uqam.ca/coursv3/pluginfile.php/552352/mod_resource/content/1/Pre%CC%81sentation%20-%20MET5311%20H15%20Cours%201.pdf}
\end{thebibliography}
\end{document}